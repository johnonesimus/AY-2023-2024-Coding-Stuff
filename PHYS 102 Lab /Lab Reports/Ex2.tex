\documentclass[10pt,a4paper,twoside]{article}
% The following LaTeX packages must be installed on your machine: amsmath, authblk, bm, booktabs, caption, dcolumn, fancyhdr, geometry, graphicx, hyperref, latexsym, natbib

\input{spp.dat}



\begin{document}

\title{\TitleFont Equipotential and Electric Field Lines Kineme}



\author[*\negthickspace]{John Onesimus G. ~Ancheta}
\affil[ ]{Institute of Mathematical Sciences and Physics, University of the Philippines Los Baños,  Philippines}
\affil[*]{\corremail{jgancheta@up.edu.ph} }

\begin{abstract}
\noindent
Perspiciatis unde omnis iste natus error sit voluptatem accusantium doloremque laudantium, totam rem aperiam, eaque ipsa quae ab illo inventore veritatis et quasi architecto beatae vitae dicta sunt explicabo. Nemo enim ipsam voluptatem quia voluptas sit aspernatur aut odit aut fugit, sed quia consequuntur magni dolores eos qui ratione voluptatem sequi nesciunt. Neque porro quisquam est, qui dolorem ipsum quia dolor sit amet, consectetur, adipisci velit, sed quia non numquam eius modi tempora incidunt ut labore et dolore magnam aliquam quaerat voluptatem. Ut enim ad minima veniam, quis nostrum exercitationem ullam corporis suscipit laboriosam, nisi ut aliquid ex ea commodi consequatur? Quis autem vel eum iure reprehenderit qui in ea voluptate velit esse quam nihil molestiae consequatur, vel illum qui dolorem eum fugiat quo voluptas nulla pariatur?

\keywords{keyword 1, keyword 2}

\end{abstract}

\maketitle
\thispagestyle{titlestyle}


%--------------------------------------------------
% the main text of your paper begins here
%--------------------------------------------------
\section{Introduction}\label{sec:intro}

Clearly states the purpose and objectives of the experiment.
Provides relevant background information and theory.
States the hypothesis or research question.
Highlights the significance of the experiment.

Many physical phenomena are modelled through vector fields. In our case, a charge creates an electric field around it, which exerts a force on another charge that enters that field. We probe this interaction
through electric potentials.

\section{Methodology}\label{sec:methods}
Lists all materials and equipment used in the experiment.
Describes the experimental procedure in a step-by-step manner.
Includes any safety precautions taken during the experiment.
Sufficient detail to allow another researcher to replicate the experiment.

To test the properties of Fields and Equipotential Lines, we needed a low voltage, electrodes,, trough, water. We place the electrodes inside the trough
with saline water. 

We utilized various set-ups such as:\\
1. When two parallel, aluminum, linear electrodes\\
2. One parallel, aluminin, linear electrode with no counterpart\\
3. Two parallel, linear electrodes with a circular conductor in the middle\\
4. Perpendicular?


\section{Results and Discussion}\label{sec:results}


\section{Conclusions}\label{sec:conclusions}
Lists all materials and equipment used in the experiment.
Describes the experimental procedure in a step-by-step manner.
Includes any safety precautions taken during the experiment.
Sufficient detail to allow another researcher to replicate the experiment.

\section*{*Additional Reminders}
\begin{itemize}
    \item The most common cause of manuscript processing delays is incorrect formatting of the (1) author block and affiliation bylines, and (2) references. Please verify that all blue web hyperlinks resolve correctly.
    \item The Full Proceedings is also prepared in print. Please follow the maximum page limit of Four (4) pages \textit{including references}. 
\end{itemize}

% Please use the style file spp-bst.bst. If you wish to use BibTeX, kindly use us the filename bibfile.bib for your bib file.
\bibliographystyle{spp-bst}
\bibliography{bibfile}

\end{document}